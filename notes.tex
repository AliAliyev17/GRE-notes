\documentclass[paper=a4]{article}
\usepackage[margin=2cm]{geometry}
\usepackage{titlesec}
\usepackage{titling}

\titleformat{\section}
{\huge\bfseries\centering}
{}
{0em}
{}[\titlerule]

\titleformat{\subsection}
{\bfseries\LARGE\centering}
{}
{0em}
{}

\titleformat{\subsubsection}
{\bfseries\large}
{}
{0em}
{}


\renewcommand{\maketitle}{
\begin{center}
	{\huge\bfseries\thetitle\\
	\em\large\bfseries\theauthor\\
	\thedate
	}
\end{center}
}

\begin{document}

\author{Ali Aliyev}
\date{\today}
\title{GRE notes}
\maketitle

\section{Reading Comprehension}

\subsection{Question Types}

\subsubsection{Global}
About passage as a whole (e.g. purpose, tone of author)
\subsubsection{Detail}
Identify specific detail from passage - ``according to the author'', ``is mentioned in the passage''
\subsubsection{Inference}
Answer that must be true based on the passage - ``suggests'', ``implies'', ``most likely agrees''
\subsubsection{Logic}
Describe why author included certain word or phrase - ``in order to'', ``primarily serves to''
\subsubsection{Vocab-in-Context}
Identify the meaning of the word used in the passage
\subsubsection{Reasoning}
Identify, point out a flaw in, stengthen or weaken author's reasoning in an argument



\subsection{Reasoning Questions}
\begin{itemize}
\item argument=evidence+conclusion
\item conclusion is a prediction, separated with key word (e.g. \em therefore)
\end{itemize}

\subsubsection{Scope Shift arguments}
\begin{itemize}
	\item Discusses different subject in conclusion
	\item Attack assumption that connects cause to result (i.e. emotional distress $\rightarrow$ anxiety)
	\item To weaken - look for answer that \textbf{refutes} relation between cause and result.
	\item To strengthen - look for answer that \textbf{supports} relation between cause and result.
\end{itemize}

\subsubsection{Representativeness arguments}
\begin{itemize}
	\item Assuming subjects in evidence (e.g. survey) represent the target group in conclusion.
	\item Information about one group used to draw conclusion about bigger group.
	\item To weaken - find argument that says two groups are \textbf{different}.
	\item To strengthen - find argument that states two groups are \textbf{similar}.
\end{itemize}

\subsubsection{Causal arguments}
\begin{itemize}
	\item Confuse correlation with causation.
	\item Bad cases decreased so new system is working.
	\item Failure to consider effects of other parameters on final result.
\end{itemize}

\end{document}
